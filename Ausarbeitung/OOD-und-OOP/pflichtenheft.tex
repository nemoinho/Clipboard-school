\documentclass[a4paper,11pt,abstracton,titlepage]{scrartcl}
\usepackage[T1]{fontenc}
\usepackage[utf8]{inputenc}
\usepackage{lmodern}
\usepackage[ngerman]{babel}
\usepackage{eurosym}
\usepackage{textcomp}
\usepackage{graphicx}
\usepackage{geometry}
\usepackage{pdfpages}

\usepackage{helvet}
\renewcommand{\familydefault}{\sfdefault}

% die Vorgaben der Prüfungsarbeit... etwas hässlich
\geometry{a4paper,left=25mm,right=25mm, top=3cm, bottom=3cm}
%\parindent 80pt

\setlength{\parskip}{10em}
\title{Clipboard-Manager\\Pflichtenheft}
\author{Tobias Schult \\  Simon Sperling \and  Felix Nehrke \\  Wojciech
Rydzewski} 
\date{15. Mai 2012}

\begin{document}

\maketitle

\tableofcontents
\thispagestyle{empty}
\newpage
% die Vorgaben der Prüfungsarbeit... etwas hässlich
\setlength{\parskip}{1em}
\setcounter{page}{1}


\section{Zielbestimmungen} 
Das Ziel ist es, eine Software in Java zur Überwachung und Manipulation des
Zwischenspeichers zu entwickeln.  Dazu wird eine Dokumentation erstellt und am
29. Mai 2012 durch den CEO der Bytehawks SE an die G18 überreicht.  Die
Dokumentation enthält genaue Beschreibungen über die Entwicklung der Software,
das objektorientierte Design und eine Funktionsbeschreibung des Programms.  Es
wird am 25. Mai 2012 eine Entwicklerpräsentation durch die Entwickler der
Bytehawks SE abgehalten, dabei wird die Entwicklung und Funktionsweise
vorgestellt. Am 29. Mai 2012 findet eine Messe statt in der das Produkt
präsentiert wird.

\subsection{Grenzkriterien} 
Für die Entwicklung muss Java\texttrademark  6 verwendet werden. Ein
Programmupgrade für Java \texttrademark 7 wird auf Nachfrage zur Verfügung
gestellt.  Die Grundfunkitonalität besteht aus Lesen und Schreiben des
Zwischenspeichers sowie Speichern des Verlaufs.  Die einzelnen Parser werden
dynamisch nach Bedarf geladen und verwendet.  Unter dem Abschnitt
Wunschkriterien werden die Parser näher erläutert.  

\subsection{Wunschkriterien}
Die Software wird durch Plugins wie z.B. Parser erweiterbar sein. Dazu werden
im Standardpaket Parser mitgeliefert. Im Standardpaket sind folgende Parser
enthalten: \newline \begin{itemize} \item RegEx-Parser (Parser zum definieren
eigener RegEx Pattern) \item ROT13-Parser (Parser zum de-/codieren von ROT13)
\end{itemize} \newline Jeder Parser bekommt Input und gibt den Output nach dem
EVA (Eingabe Verarbeitung Ausgabe) Prinzip aus. Ein Beispiel dazu: der Parser
bekommt die Eingabe "`Hallo Welt"' nun soll der Parser diese Eingabe ROT13
verschlüsseln.  Als Ausgabe gibt der Parser "`Unyyb Jryg"' zurück.  Man kann in
den Einstellungen die jeweiligen Parser auswählen.

\subsection{Abgenzungskriterien}
Das Programm erfüllt nur die Funktionalitäten welche beschrieben sind.  Durch
das Pluginssystem, lassen sich später nachträglich, neue Parser implementieren
oder vorhandene können ausgetauscht/aktualisiert werden.  

\section{Produkteinsatz}
Das Einsatzgebiet dieser Software ist überwiegend bei Entwicklern angesiedelt.
Die Software vereinfacht das Manipulieren von Zwischenspeichereinträgen. Dennoch
kann das Produkt auch privat oder geschäftlich genutzt werden.  Ein Beispiel für
ein privates Nutzungsszenario könnte folgendermaßen aussehen: Jemand kopiert von
Word den Text in den Zwischenspeicher, dieser soll verschlüsselt an eine andere
Person geschickt werden. Beim Kopieren in den Zwischenspeicher kommt das Produkt
zum Einsatz und verschlüsselt den Eintrag. Beim Einfügen in die E-Mail wird das
System nun die verschlüsselte Nachricht zum Einfügen verwenden. Außerdem kann
man, wenn es der Parser zulässt, diesen konfigurieren.

\subsection{Anwendungsbereiche}
Ein Anwendungsbereich wird in der Entwicklung oder im Vertrieb sein, um das
Kopieren von formatierten Text sehr zu vereinfachen.  

\subsection{Zielgruppe}
Zielgruppe für dieses Produkt sind Personen die viel Text kopieren müssen und
dabei jedesmal Formatierung löschen oder ähnliche Veränderungen vornehmen
müssen.  

\section{Produktkonfiguration}
Für das Produkt wird Java\texttrademark  6 benötigt. Durch den Einsatz von
Java\texttrademark  ist dieses Programm theoretisch auf Betriebssystemen mit
Java\texttrademark  6 lauffähig.  

\section{Benutzerschnittstelle}
Die Benutzerschnittstelle besteht aus einer Konfigurationsoberfläche.  In dieser
Oberfläche kann der Benutzer seinen Zwischenspeicherverlauf sehen und die
einzelenen Profile können eingestellen. In den Profilen können die zu
verwendenden Parser ausgewählt werden. 

\section{Qualitäts-Zielbestimmungen}
Das Programm kann den Zwischenspeicher anzeigen und durch verschiedene Parser
verändern. Diese Parser können frei definiert werden. Desweiteren sind in der
Auslieferung zwei funktionsfähige Parser enthalten.

\section{Globale Testfälle} 
Die Testfälle betreffen das Grundprogramm und die Standard-Parser.  Es wird bei
den Tests darauf geachtet, dass die Funktionalität sichergestellt ist.  Diese
Tests werden auf Windows und auf Linux durchgeführt.


\section{Entwicklungs-Konfiguration}
Die Entwicklungskonfiguration besteht aus meheren Linux-Systemen und einem
Windows-System. Diese Systeme verwenden OpenJDK und die offizielle Oracle  JDK.

\section{Ergänzungen}
Die Software ist in mehrere Teile unterteilt zum Einen den Kernel (Core) mit den
Grundfunktionalitäten, d.h. lesen/schreiben des Zwischenspeichers, zum Anderen
das Plugin System welches die Funktionalität beliebig erweitert. Aufgebaut auf
den Core ist die GUI nach der Schichtenarchitektur. Folgendes Beispiel zeigt die
Schichtenarchitektur.
\newline \begin{figure}[htbp]
\centering
\includegraphics[scale=0.7]{Schichtenarchitektur} 
\caption{Bild der geplanten Schichtenarchitektur} 
\end{figure} 
\newpage 

\section{Unterschriften}
Mit Unterschrift dieses Pflichtenheft stimmen beide
Parteien den, in diesem Dokument aufgeführten Bedingungen und Voraussetzungen
zu.  
\newline\newline 
\_\_\_\_\_\_\_\_\_\_\_\_\_\_\_\_\_\_\_\_\_\_\_\_\_\_\_\_
\newline Auftraggeber
(G18) 
\newline \newline\newline
\_\_\_\_\_\_\_\_\_\_\_\_\_\_\_\_\_\_\_\_\_\_\_\_\_\_\_\_
\newline Auftragnehmer (Bytehawks SE)
\newline 

\end{document}
